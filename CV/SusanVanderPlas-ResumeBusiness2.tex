\documentclass[10pt]{tccv}
\usepackage[english]{babel}
\usepackage[nosort]{cite}
\usepackage{xstring} % bold individual name in bib entries
% \usepackage{biblatex} % separate bibliographies
\nocite{*}
\def\FormatMaidenName#1{%
  \IfSubStr{#1}{Koons}{\textbf{#1}}{#1}%
}
\def\FormatName#1{%
  \IfSubStr{#1}{VanderPlas}{\textbf{#1}}{\FormatMaidenName{#1}}%
}
% Color combination: 0099cc, ccffcc, 66ccff, 003399


\definecolor{titlecolor}{HTML}{003399}
\definecolor{link}{HTML}{0099CC}

\hypersetup{colorlinks, breaklinks, 
            linkcolor=link, 
            urlcolor=link, 
            citecolor=link}
 


\makeatletter
\renewenvironment{thebibliography}[1]
     {\list{\@biblabel{\@arabic\c@enumiv}}%
           {\settowidth\labelwidth{\@biblabel{#1}}%
            \leftmargin\labelwidth
            \advance\leftmargin\labelsep
            \@openbib@code
            \usecounter{enumiv}%
            \let\p@enumiv\@empty
            \renewcommand\theenumiv{\@arabic\c@enumiv}}%
      \sloppy
      \clubpenalty4000
      \@clubpenalty \clubpenalty
      \widowpenalty4000%
      \sfcode`\.\@m}
     {\def\@noitemerr
       {\@latex@warning{Empty `thebibliography' environment}}%
      \endlist}
\makeatother
\begin{document}

\part{Susan VanderPlas}

\personal
    [www.srvanderplas.github.io]
    {802 17th St, Auburn, NE 68305}
    {(515) 509-6613}
    {srvanderplas@gmail.com}
\changeurlcolor{link}

\section{Skills and Strengths}
\begin{description}[leftmargin=*]
\item [Statistical modeling] to make predictions and decisions based on available information
\item [Risk assessment,] using data to understand probable outcomes, assess market changes, and identify opportunities
\item [Optimization] Identifying targets for process improvement and sources of variability
\item [Communicating and summarizing information] with written reports and well-designed graphics
\end{description}

\section{Education}
\begin{yearlist}
\item{2011-2015*}
     {Ph.D. in Statistics}
     {\begin{minipage}{.32\textwidth}\textsc{\small Perception \& Statistical Graphics}\\ Iowa State University\end{minipage}}
\item{2009-2011}
     {M.S. in Statistics}
     {Iowa State University}
\item{2005-2009}
     {B.S. in Psychology and Applied Mathematical Sciences}
     {Texas A\&M University}
\end{yearlist}

\section{Technical Skills}
\begin{description}[leftmargin=*]
\item [Statistical Techniques]
Linear, generalized, mixed, and hierarchical models. Data mining, Bayesian, time series, and nonparametric analysis.\vspace{-6pt}
\item [Statistical Software]
Expert R user, SAS (linear and mixed models), JMP. \vspace{-6pt}
\item [Programming and Database Software]
C and C++, JavaScript, git, SQL and MySQL. \vspace{-6pt}
\item [Web Development]
Interactive applet development with Shiny, d3 interactive graphics, use of knitr and pandoc to automate report generation, Apache web server administration.\vspace{-6pt}
\item [Computer Skills] Proficient in Microsoft Office.\\Familiar with Windows and Linux. 
\end{description}

\section{Awards}
\begin{factlist}[leftmargin=.1cm,itemindent=.1cm,labelwidth=\itemindent,labelsep=.1cm]\itemsep-2pt
\item[ASA Student Paper Award (Graphics)] 2013
\item[NSF IGERT Fellowship] 2009-2011
\item[Texas A\&M] Foundation, University, Liberal Arts, Psychology, and Mathematics Honors
% \item[Ugrad. Research Fellow] Texas A\&M, 2009
\item[University Scholar] Texas A\&M, 2006-2009
% \item[Astronaut Scholar] 2008-2009
% \item[President's Endowed Scholarship] 2005-2009
% \item[Director's Excellence Award] 2005-2009
% \item[National Merit Award] Texas A\&M
% \item[National Merit Scholar] 2005
\end{factlist}


\section{Experience}
\begin{eventlist}
% \item{Date}
%      {Company}
%      {Role}
% 
% Description

\item{Summer 2012-Present}{Ph.D. Research, ISU}{Statistical Visualization}{
Modeled effectiveness of graphical designs for accurate communication of statistical results. 
% Designed and analyzed experiments to understand human perception of statistical graphics. Optimized graphics to clearly communicate statistical results and counteract perceptual biases identified during experiments. Explored the hierarchy of graphical features, how visual reasoning abilities relate to the ability to understand charts, and the occurrence of optical illusions in statistical graphics. \cite{sineillusionjcgs,jsm2014,jsm2014userpanel,jsm2013}
}

\item{Fall 2013-Present}{USDA and ISU Statistics}{Soybean Genome Analysis}{
Identified important features of soybean genetic data, including genes which contribute to disease resistance and increased yield. 
% Created interactive applets and visualizations to communicate results to biologists; collected soybean parentage information into a searchable database for future analysis.
% Analyzed large quantities of soybean genetics data to identify inheritance, important genes, and copy number variation. Created interactive applets presenting the data along with graphics designed to encourage biologists to explore the results. Assembled a database of known soybean parentage to facilitate further research and wrote code to efficiently search the database to identify the lineage of any variety in the database. }
}

\item{Fall 2012-Present}{Nebraska Public Power}{Statistical Consulting}{
Accurately predicted the number of maintenance outages that occurred during the first 24-month cycle using 18-month cycle data. Assembled a database of power prices from several regions to examine the financial impact of maintenance scheduling and explore the conditions leading to negative power prices.
}

\item{Spring 2013-Present}{\phantom{ISU}}{R Course Instructor\vspace{-14pt}}{
Designed and conducted workshops to teach R skills to members of the university and local business community. 
% Workshop topics included an introduction to R, linear models, graphics, data management and cleaning, package development, automated report generation, and creation of interactive web applets.
% ggplot2, data management with plyr, reshape2, and stringr, package development, document creation with knitr, linear models, and creating web applets with Shiny. 
}

\item{2013-2014}{\phantom{ISU}}{Statistics Education Applets\vspace{-14pt}}{
Created web-based applets to teach statistical techniques interactively. 
% Applets include Method of Least Squares, ANOVA, K Means, Regression diagnostics, and other introductory statistics concepts. 
Link: \href{http://vanderplas.dyndns-remote.com:3838/}{Applets}}
% 
% \item{Fall 2013}{\phantom{ISU}}{Modeling Student Learning\vspace{-14pt}}{Provided modeling advice and statistical expertise to aerospace engineering professsors conducting research on active learning.}

\item{Summer 2013-14}{\phantom{R Project}}{Google Summer of Code}{
Worked to develop the \texttt{animint} package for R to translate R graphics into d3 interactive JavaScript graphics. Participated in the project in 2013, and returned to serve as a mentor for the project in 2014.}

% \item{Fall 2011 - Spring 2013}{ISU Statistics}{Teaching Assistant}{
% Demonstrated statistical methodology to undergraduate and graduate students in business, biology, social sciences, and engineering
% %along with the use of statistical software (R, SAS, and JMP)
% .}

\item{Jan-Aug 2012}{Iowa DOT and ISU Statistics}{Modeling Collisions and Road Design}{
Modeled effectiveness of road interventions on traffic accidents and fatalities. 
% Identified an error in the recorded data that affected prior analyses and developed a method to adapt the analysis to account for the error. 
}

\item{2010-2011}{MS Research, ISU}{Nonparametric Peak Identification}{
Worked with the materials science and engineering department at ISU to develop and implement nonparametric methods for peak detection in mass spectroscopy data.
Helped to fit systems of differential equations to spectroscopy data in order to extract additional information about the atomic structure of the material. 
}

% \item{Summer 2009}{Iowa State University}{NSF Research Experience for Undergrads}{
% Worked with biologists and bioinformaticians to compare homologous gene expression in humans, pigs, and mice.\cite{towfic2010detection}}  
% 
% \item{Summer 2008}{University of Nebraska}{NSF Research Experience for Undergrads}{
% Created a mathematical model describing electrical impulse transmission and decay along neurons with varying states of myelination.}
\end{eventlist}
% 
% \vspace{4.5in}
% \section{Publications \& Presentations}
% \bibliographystyle{myunsrt}
% \bibliography{papers,presentations}
% \nocite{*}

\end{document}

% 
% \section{Public projects}
% \begin{yearlist}
% 
% \item{2013}
%      {silverstripe (\href{http://silverstripe.entidi.com/}{silverstripe.entidi.com})}
%      {Themes and extensions for SilverStripe}
% 
% \end{yearlist}

% \section{Communication skills}
% \begin{factlist}
% % \item{Italian}{Native speaker}
% 
% \end{factlist}
