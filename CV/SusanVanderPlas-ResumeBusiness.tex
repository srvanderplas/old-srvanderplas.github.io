%%% LaTeX Template: Designer's CV
%%%
%%% Source: http://www.howtotex.com/
%%% Feel free to distribute this template, but please keep the referal to HowToTeX.com.
%%% Date: March 2012


%%%%%%%%%%%%%%%%%%%%%%%%%%%%%%%%%%%%%
% Document properties and packages
%%%%%%%%%%%%%%%%%%%%%%%%%%%%%%%%%%%%%
\documentclass[letterpaper,12pt,final]{memoir}

% misc
\renewcommand{\familydefault}{bch}  % font
\pagestyle{empty}					% no pagenumbering
\setlength{\parindent}{0pt}			% no paragraph indentation


% required packages (add your own)									% column layout
\usepackage[top=1cm,left=1cm,right=1cm,bottom=1cm]{geometry}% margins
\usepackage{graphicx}										% figures
\usepackage{url}											% URLs
\usepackage[unicode]{hyperref}
\usepackage[usenames,dvipsnames]{xcolor}					% color
\usepackage{multicol}										% columns env.
	\setlength{\multicolsep}{0pt}
\usepackage{paralist}										% compact lists
\usepackage{tikz}
\usepackage{amsmath,amssymb}
\usepackage{needspace} % Requires X lines for breaking a column/page

\hypersetup{colorlinks, breaklinks, 
            linkcolor=RoyalBlue, 
            urlcolor=RoyalBlue, 
            citecolor=RoyalBlue}
            
%%%%%%%%%%%%%%%%%%%%%%%%%%%%%%%%%%%%%
% define macros (for convience)
%%%%%%%%%%%%%%%%%%%%%%%%%%%%%%%%%%%%%
\newcommand{\Sep}{\vspace{1.5em}}
\newcommand{\MedSep}{\vspace{1em}}
\newcommand{\SmallSep}{\vspace{0.5em}}

\newenvironment{AboutMe}
	{\ignorespaces\textbf{\color{RoyalBlue} About me}}
	{\MedSep\ignorespacesafterend}
	
\newcommand{\CVSection}[1]
	{\Large\textbf{#1}\par
	\SmallSep\normalsize\normalfont}

\newcommand{\CVItem}[1]
	{\textbf{\color{RoyalBlue} #1}}

\newcommand{\Experience}[3]{\parbox{\linewidth}{\CVItem{#1\hfill#2}\\*{\small #3\MedSep}}}


%%%%%%%%%%%%%%%%%%%%%%%%%%%%%%%%%%%%%
% Begin document
%%%%%%%%%%%%%%%%%%%%%%%%%%%%%%%%%%%%%
\begin{document}

% Left frame
%%%%%%%%%%%%%%%%%%%%
% \begin{figure}
% 	\hfill
% 	\includegraphics[width=0.6\columnwidth]{photo}
% 	\vspace{-7cm}
% \end{figure}

% \begin{flushright}\small
% 	Susan R. VanderPlas \\
% 	{\hspace{-24pt}\url{srvanderplas@gmail.com}} \\
% 	\url{www.github.com/srvanderplas} \\
% 	(515) 509-6613
% \end{flushright}\normalsize
% \framebreak


% Right frame
%%%%%%%%%%%%%%%%%%%%
\begin{minipage}[t]{.6\linewidth}
\Huge\bfseries {\color{RoyalBlue} Susan Vanderplas} \\
\Large\bfseries  Data Scientist 
\end{minipage}\hfill
\begin{minipage}[t]{.35\linewidth}
\begin{flushright}\small
\phantom{\Huge SVP}
802 17th St., Auburn, NE 68305 \\
(515) 509-6613\\
\url{srvanderplas@gmail.com} 
\end{flushright}
\end{minipage}

\normalsize\normalfont
\SmallSep
% About me
\begin{AboutMe}
I am a data scientist, that is, I transform data into informed decisions by building and interpreting mathematical models. 
I work with subject matter experts to understand and quantify prior knowledge, then incorporate data to create robust, accurate predictive models. 
I offer well-rounded statistical skills, programming expertise, and experience communicating statistical information to those outside the field. 
\end{AboutMe}

% Education
\CVSection{Education}
%% Use minipage to get unequal width columns
\begin{minipage}[t]{.35\linewidth}
\CVItem{Iowa State University}\\
2015 - Ph.D. in Statistics; GPA 3.71\\
2011 - M.S. Statistics; GPA 3.69
\end{minipage}\hfill
\begin{minipage}[t]{.5\linewidth}
\CVItem{Texas A\&M University}\\
2009 - B.S. Psychology and Applied Math; GPA 3.88
\end{minipage}\hfill
\MedSep

% Skills
\CVSection{Skills}
\CVItem{Statistical Techniques}
\begin{multicols}{3}
\begin{compactitem}[\color{RoyalBlue}$\bullet$]
	\item Estimation with error bands
  \item Prediction
	\item Risk assessment
  \item Reliability analysis
	\item Time-series models
	\item Bayesian methods
  \item Nonparametric statistics
	\item Data mining
	\item Experimental design
\end{compactitem}
\end{multicols}
\SmallSep
\CVItem{Computer Skills}
\begin{multicols}{3}
\begin{compactitem}[\color{RoyalBlue}$\bullet$]
	\item R (statistical programming)
  \item SAS statistical software
	\item Data dashboard design
	\item C, C++
	\item JavaScript
	\item SQL/MySQL database
  \item Web server administration
  \item MS Office
\end{compactitem}
\end{multicols}
\MedSep 

\CVSection{Experience}
\begin{multicols}{2}

\Experience{Engineering Statistician}
{Aug. 2015-present}
{Consulted on engineering and business decisions for Nebraska Public Power District and Cooper Nuclear Station.}

\Experience{Statistical Visualization Research}
{2012-2015}
{Modeled effectiveness of graphical designs for accurate communication of statistical results.}

\Experience{USDA Soybean Genome Project}
{2013 - 2015}
{Identified important features of soybean genetic data, including genes which contribute to disease resistance and increased yield. Created dynamic reports, interactive data dashboards, and other tools to communicate results effectively.}

\Experience{Google Summer of Code}
{Summers 2013-15}
{Worked to develop the \texttt{animint} package for R, translating R graphics into d3 interactive JavaScript graphics. Participated in the project in 2013, and returned to serve as a mentor for the project in 2014 and 2015.}

\Experience{R Course Instructor}
{2013 - 2015}
{Designed and conducted workshops to teach statistical computing to members of the university and local business community.}

% \Experience{Statistics Education Applets}
% {2013 - 2014}
% {Created web-based applets to teach statistical techniques interactively. 
% Link: \href{http://vanderplas.dyndns-remote.com:3838/}{Applets}}

\Experience{Industrial Statistics}
{2012 - 2015}
{Served as an informal consultant to a public utility. Accurately predicted the number of plant outages that occurred during the first 24-month plant cycle using 18-month cycle data. }

\Experience{Iowa Department of Transportation}
{2012}
{Examined the effect of road layout and construction on driver safety (collisions, fatalities).}

\Experience{Materials Science Collaboration}
{2010-2011}
{Increased accuracy and efficiency of peak detection (vs. manual identification) using robust quantile analysis. }
\end{multicols}
%%%%%%%%%%%%%%%%%%%%%%%%%%%%%%%%%%%%%
% End document
%%%%%%%%%%%%%%%%%%%%%%%%%%%%%%%%%%%%%
\end{document}
