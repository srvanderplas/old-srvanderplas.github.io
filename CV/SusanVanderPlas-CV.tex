%% start of file `template.tex'.
%% Copyright 2006-2013 Xavier Danaux (xdanaux@gmail.com).
%
% This work may be distributed and/or modified under the
% conditions of the LaTeX Project Public License version 1.3c,
% available at http://www.latex-project.org/lppl/.


\documentclass[11pt,letterpaper,sans,unicode]{moderncv}        % possible options include font size ('10pt', '11pt' and '12pt'), paper size ('a4paper', 'letterpaper', 'a5paper', 'legalpaper', 'executivepaper' and 'landscape') and font family ('sans' and 'roman')

\usepackage[nosort]{cite}
\usepackage{xstring} % bold individual name in bib entries
% \usepackage{biblatex} % separate bibliographies
\nocite{*}
\def\FormatMaidenName#1{%
  \IfSubStr{#1}{Koons}{\textbf{#1}}{#1}%
}
\def\FormatName#1{%
  \IfSubStr{#1}{VanderPlas}{\textbf{#1}}{\FormatMaidenName{#1}}%
}
% Color combination: 0099cc, ccffcc, 66ccff, 003399

% character encoding
\usepackage[utf8]{inputenc}                       % if you are not using xelatex ou lualatex, replace by the encoding you are using

% moderncv themes
\moderncvstyle{classic}                             % style options are 'casual' (default), 'classic', 'oldstyle' and 'banking'
\moderncvcolor{blue}                               % color options 'blue' (default), 'orange', 'green', 'red', 'purple', 'grey' and 'black'
%\renewcommand{\familydefault}{\sfdefault}         % to set the default font; use '\sfdefault' for the default sans serif font, '\rmdefault' for the default roman one, or any tex font name
%\nopagenumbers{}                                  % uncomment to suppress automatic page numbering for CVs longer than one page


%\usepackage{CJKutf8}                              % if you need to use CJK to typeset your resume in Chinese, Japanese or Korean

% adjust the page margins
\usepackage[scale=0.75]{geometry}
\setlength{\hintscolumnwidth}{2.3cm}                % if you want to change the width of the column with the dates
%\setlength{\makecvtitlenamewidth}{10cm}           % for the 'classic' style, if you want to force the width allocated to your name and avoid line breaks. be careful though, the length is normally calculated to avoid any overlap with your personal info; use this at your own typographical risks...

% Timeline
\usepackage{moderntimeline}
\tlmaxdates{2005}{2016} % Set the scale
\tlwidth{0.8ex} % Set the line width - space under the top label is 1pt more
\tltext{\tiny} % set label text size

\usepackage[unicode]{hyperref}
\usepackage{xcolor}
\definecolor{link}{HTML}{3873B3}
\hypersetup{colorlinks, breaklinks, 
            linkcolor=link, 
            urlcolor=link, 
            citecolor=link}
            
% personal data
\name{Susan}{VanderPlas}
% \title{Resumé title}                               % optional, remove / comment the line if not wanted
\address{802 17th St.}{Auburn, NE 68305}{}% optional, remove / comment the line if not wanted; the "postcode city" and "country" arguments can be omitted or provided empty
\phone[mobile]{(515) 509-6613}                   % optional, remove / comment the line if not wanted; the optional "type" of the phone can be "mobile" (default), "fixed" or "fax"
% \phone[fixed]{+2~(345)~678~901}
% \phone[fax]{+3~(456)~789~012}
\email{srvanderplas@gmail.com}                               % optional, remove / comment the line if not wanted
% \homepage{www.srvanderplas.github.io}                         % optional, remove / comment the line if not wanted
% \social[linkedin]{john.doe}                        % optional, remove / comment the line if not wanted
% \social[twitter]{jdoe}                             % optional, remove / comment the line if not wanted
\social[github]{srvanderplas}                              % optional, remove / comment the line if not wanted
% \extrainfo{additional information}                 % optional, remove / comment the line if not wanted
% \photo[64pt][0.4pt]{picture}                       % optional, remove / comment the line if not wanted; '64pt' is the height the picture must be resized to, 0.4pt is the thickness of the frame around it (put it to 0pt for no frame) and 'picture' is the name of the picture file
% \quote{Some quote}                                 % optional, remove / comment the line if not wanted

% to show numerical labels in the bibliography (default is to show no labels); only useful if you make citations in your resume
%\makeatletter
%\renewcommand*{\bibliographyitemlabel}{\@biblabel{\arabic{enumiv}}}
%\makeatother
%\renewcommand*{\bibliographyitemlabel}{[\arabic{enumiv}]}% CONSIDER REPLACING THE ABOVE BY THIS

% bibliography with mutiple entries
% \usepackage{multibib}
% \newcites{papers,presentations}{{Publications},{Presentations}}


%----------------------------------------------------------------------------------
%            content
%----------------------------------------------------------------------------------
\begin{document}
%-----       resume       ---------------------------------------------------------
\makecvtitle

\section{Education}
\tllabelcventry{2005}{2009}{2005--2009}{Bachelor of Science}{Texas A\&M University}{}{}{Major: Psychology and Applied Mathematical Sciences (Statistics), Minor: Neuroscience}  % arguments 3 to 6 can be left empty
\tllabelcventry{2009}{2011}{2009--2011}{Master of Science in Statistics}{Iowa State University}{}{}{Creative Component: Nonparametric statistical analysis of Atom Probe Tomography spectra\\ Chair: Dr.~Alyson Wilson, Committee Members: Dr.~Alicia Carriquiry, Dr.~Krishna Rajan}
\tllabelcventry{2011}{0}{2011--15}{Doctor of Philosophy in Statistics}{Iowa State University}{}{}{}

\subsection{Dissertation}
\cvitem{Title}{\emph{The Perception of Statistical Graphics}}
\cvitem{Committee}{Dr.~Heike Hofmann (Chair), Dr.~Dianne Cook, Dr.~Sarah Nusser, Dr.~Max Morris, Dr.~Erin McDonald, Dr.~Stephen Gilbert}
\cvitem{Abstract}{\small Research on statistical graphics and visualization generally focuses
on new types of graphics, new software to create graphics, interactivity, and usability
studies. Our ability to interpret and use statistical graphics hinges on the interface between
the graph itself and the brain that perceives and interprets it, and there is substantially less
research on the interplay between graph, eye, brain, and mind than is sufficient to understand
the nature of these relationships. This dissertation further explores the interplay between a static
graph, the translation of that graph from paper to mental representation (the journey from
eye to brain), and the mental processes that operate on that graph once it is transferred into
memory (mind). Understanding the perception of statistical graphics will allow researchers
to create more effective graphs which produce fewer distortions and viewer errors while reducing
the cognitive load necessary to understand the information presented in the graph.
}

\section{Research}
\tltextstart[base]{\tiny}
% \tltextend[base]{\tiny}
% \tlcventry{year1}{year2}{Job Title}{Employer}{City}{}{Description}


\tllabelcventry{2015.4}{0}{May 2015}{Postdoc, Office of the Vice Pres. for Research}{Iowa State University}{Ames, IA}{}{
\begin{itemize}
\item Evaluated faculty funding start-up packages to explore links between start-up funding and future productivity.
\item Explored natural variation and underlying trends in grant receipts across Iowa State over a 20 year period. 
\end{itemize}}

\tllabelcventry{2012}{2015.6}{2012 - Jun 2015}{PhD Research}{Iowa State University}{}{Ames, IA}{Designed and analyzed experiments to understand human perception of statistical graphics and optimized graphics to clearly communicate statistical results. \cite{jsm2014userpanel}%
\begin{itemize}%
\item The \href{https://github.com/heike/sine-illusion}{Sine Illusion} in Statistical Graphics: How does this common illusion effect the information we take in from graphs? \cite{sineillusionjcgs,jsm2014,jsm2013}
  \begin{itemize}%
    \item Won the ASA Student Paper Award (2014) for the Graphics Section (\href{https://github.com/srvanderplas/LieFactorSine}{Paper})
    \item Created Shiny applets to \href{http://glimmer.rstudio.com/srvanderplas/SineIllusion/}{demonstrate the illusion} and \href{http://glimmer.rstudio.com/srvanderplas/SineIllusionShiny/}{test it's effect}. 
  \end{itemize}
\item Statistical Graphics and Visual Aptitude: How are spatial reasoning abilities related to the ability to read statistical graphics? (\href{https://github.com/srvanderplas/VisualAptitude}{Paper Draft}). 
\item Hierarchy of Graphical Features: Which features of statistical graphics dominate the perceptual experience? Do outliers matter more than trend lines? 
\end{itemize}}

\tllabelcventry{2013}{2015.8}{2013 - Aug 2015}{Research Assistant, USDA Soybean Genome Project}{Iowa State University}{Ames, IA}{}{
\begin{itemize}
\item Analyzed large quantities of soybean genetics data to identify inheritance, important genes, single nucleotide polymorphisms, and copy number variation.
\item Created interactive applets presenting the data and appropriate graphics designed to encourage exploration of the results by biologists.
\item Assembled a database of known soybean parantage to facilitate further research, and created an interactive applet to display the lineage of any variety in the database.
\end{itemize}}

\tllabelcventry{2012}{2012.8}{Jan-Aug 2012}{Research Assistant, Iowa Dept. of Transportation}{Iowa State University}{Ames, IA}{}{Developed a hierarchical Bayesian model to determine the effectiveness of road interventions on traffic accidents and fatalities. Discovered a previously unknown error in the data used in prior analyses using exploratory techniques, and developed a method to compensate for the missing data.}

\tlcventry{2010}{2011}{M.S. Research}{Iowa State University}{Ames, IA}{}{Worked with materials scientists and engineers to develop and implement nonparametric methods for automatic peak detection in mass spectroscopy data. Fit systems of differential equations to spectroscopy data based on theoretical concepts from quantum physics to facilitate inference about the atomic structure of a material.}

\tllabelcventry{2009.7}{2010}{Fall 2009}{Research Rotations in Bioinformatics}{Iowa State University}{Ames, IA}{}{Explored applications of the EM algorithm to next-generation sequencing data error detection and modeled the relationship between age and fertility in reptiles. Each project lasted about 6 weeks; rotations were structured to allow new students to explore several facets of bioinformatics, and included wet-lab and computational experiences. }

\tllabelcventry{2009.45}{2009.7}{Summer 2009}{NSF Research Experience for Undergraduates}{Iowa State University}{Ames, IA}{}{Worked with biologists and bioinformaticians to compare homologous gene expression in humans, pigs, and mice.\cite{towfic2010detection}}

\tllabelcventry{2008.45}{2008.7}{Summer 2008}{NSF Research Experience for Undergraduates}{University of Nebraska}{Lincoln, NE}{}{Created a mathematical model describing electrical impulse transmission and decay along neurons with varying states of myelination.}

\section{Teaching}
\tlcventry{2013}{0}{R Course Instructor}{Iowa State University Stat Dept.}{Ames, IA}{}{Designed and conducted workshops to teach R skills to the members of the university and local business community. Workshop topics included an introduction to R, ggplot2, data management with plyr, reshape2, and stringr, package development, document creation with knitr, linear models, and creating web applets with Shiny.}

\tllabelcventry{2013}{2013.5}{Spring 2013}{Statistical Methods for Research}{Iowa State University Stat Dept.}{Ames, IA}{}{Held office hours and graded labs and tests for Stat 401, a class composed primarily of graduate engineering students.}

\tlcventry{2012}{2013}{Introduction to Business Statistics II}{Iowa State University Stat Dept.}{Ames, IA}{}{Taught undergraduate business students statistical methods and use of JMP statistical software. Responsibilities included holding office hours and evening help sessions, developing lab materials, managing the course website on Blackboard, and grading labs, homework, and tests.}

\tllabelcventry{2011.5}{2012}{Fall 2011}{Statistical Methods for Research}{Iowa State University Stat Dept.}{Ames, IA}{}{Taught graduate social science students statistical methods and use of SAS statistical software. Responsibilities included teaching lab sessions, creating lab materials, holding office hours and grading homework and lab materials.}

\tllabelcventry{2011.5}{2012}{Fall 2011}{Empirical Methods for Comp. Sci.}{Iowa State University Stat Dept.}{Ames, IA}{}{Held office hours and graded homework for Stat 430, a class composed of graduate bioinformatics and computer science students.}


\section{Software Development}

\tlcventry{2013}{0}{Statistics Teaching Applets}{Iowa State University}{Ames, IA}{}{Created and redesigned web-based applets to teach statistical techniques interactively. Applets covered topics such as the method of least squares, ANOVA, k-means, regression diagnostics, and t-tests. (\href{http://vanderplas.dyndns-remote.com:3838/}{Applets})}

\tlcventry{2013}{2014}{Animint Developer}{R Project}{Google Summer of Code}{}{Worked to develop the \texttt{animint} package for R to translate \texttt{ggplot2} into d3 interactive JavaScript graphics. Participated in the project in 2013, adding support for all ggplot2 geoms as well as most scales and axes. Returned to serve as a mentor for the project in 2014. \cite{animintmanual,animintpaper,gprug2014}}

\section{Consulting}

\tllabelcventry{2014.8}{0}{2014}{Statistical Web Development}{Iowa State Extension Service}{Ames, IA}{}{Designed interactive web tools to assist soybean farmers with making informed decisions on planting and cultivar choices for their geographic location. Created clear, easy-to-read statistical graphics with multiple layers of detail, presented using responsive Shiny web applets. (\href{http://www.extension.iastate.edu/CropNews/2015/0416Licht.htm}{Announcement}, \href{http://agron.iastate.edu/CroppingSystemsTools/soybean-decisions.html}{Tool})}

\tllabelcventry{2012.8}{0}{2012}{Informal Consultant}{Cooper Nuclear Station}{Brownville, NE}{}{Provided informal statistical recommendations to nuclear engineers on proper methods for bootstrap, k95/95 intervals, probability analysis, and other modeling questions. Helped to estimate capacity factor using block bootstrap, answered questions about probability theory and model assessment, and assessed violations of modeling assumptions. Assembled data sets containing years of hourly power prices to explore downpower timing and market relationships.}

\tllabelcventry{2013.5}{2014}{Fall 2013}{Consultant - Aerospace Engineering}{Iowa State University}{Ames, IA}{}{Provided modeling advice and statistical expertise to aerospace engineering professsors conducting research on active learning.}



% \section{Languages}
% \cvitemwithcomment{Language 1}{Skill level}{Comment}
% \cvitemwithcomment{Language 2}{Skill level}{Comment}
% \cvitemwithcomment{Language 3}{Skill level}{Comment}
% 
\section{Technical Skills}
\cvitem{Statistical Software}{R (programming, graphics, package development, web scraping)\newline SAS (linear and mixed models)\newline JMP (basic analysis and data mining)}
\cvitem{Languages}{C and C++, JavaScript, SQL, python (for web scraping)}
\cvitem{Web Development}{Shiny (library for interactive web applets), d3 interactive graphics, \texttt{knitr} and \texttt{pandoc} for integration of code, results, and documentation, Apache and MySQL web server configuration and administration}
\cvitem{Operating Systems}{Ubuntu (system administration)\newline Windows}

% \cvdoubleitem{category 1}{XXX, YYY, ZZZ}{category 4}{XXX, YYY, ZZZ}
% \cvdoubleitem{category 2}{XXX, YYY, ZZZ}{category 5}{XXX, YYY, ZZZ}
% \cvdoubleitem{category 3}{XXX, YYY, ZZZ}{category 6}{XXX, YYY, ZZZ}

% \section{Interests}
% \cvitem{hobby 1}{Description}
% \cvitem{hobby 2}{Description}
% \cvitem{hobby 3}{Description}

\section{Awards}
\cvitemwithcomment{ASA}{Student Paper Award (Graphics)}{2013}
\cvitemwithcomment{NSF}{IGERT Fellowship}{2009-2011}
\cvitemwithcomment{Texas A\&M}{Foundation, University, Liberal Arts, Psychology, and Math Honors}{2009}
\cvitemwithcomment{Texas A\&M}{Undergraduate Research Fellow}{2009}
\cvitemwithcomment{Texas A\&M}{University Scholar}{2006-2009}
\cvitemwithcomment{Texas A\&M}{Astronaut Scholar}{2008-2009}
\cvitemwithcomment{Texas A\&M}{President's Endowed Scholarship}{2005-2009}
\cvitemwithcomment{Texas A\&M}{Director's Excellence Award}{2005-2009}
\cvitemwithcomment{Texas A\&M}{National Merit Award}{2005-2009}
\cvitemwithcomment{}{National Merit Scholar}{2005}
% \cvlistitem{Item 1}
% \cvlistitem{Item 2}
% \cvlistitem{Item 3. This item is particularly long and therefore normally spans over several lines. Did you notice the indentation when the line wraps?}
% 
% \section{Extra 2}
% \cvlistdoubleitem{Item 1}{Item 4}
% \cvlistdoubleitem{Item 2}{Item 5}
% \cvlistdoubleitem{Item 3}{Item 6. Like item 3 in the single column list before, this item is particularly long to wrap over several lines.}

% \section{References}
% \begin{cvcolumns}
%   \cvcolumn{Category 1}{\begin{itemize}\item Person 1\item Person 2\item Person 3\end{itemize}}
%   \cvcolumn{Category 2}{Amongst others:\begin{itemize}\item Person 1, and\item Person 2\end{itemize}(more upon request)}
%   \cvcolumn[0.5]{All the rest \& some more}{\textit{That} person, and \textbf{those} also (all available upon request).}
% \end{cvcolumns}

\clearpage
% Publications from a BibTeX file without multibib
%  for numerical labels: 
\renewcommand{\bibliographyitemlabel}{\@{\arabic{enumiv}}}% CONSIDER MERGING WITH PREAMBLE PART
%  to redefine the heading string ("Publications"): 
\renewcommand{\refname}{Publications and Presentations}
\nocite{*}
\bibliographystyle{myunsrt}
\bibliography{papers,presentations}                        % 'publications' is the name of a BibTeX file

% Publications from a BibTeX file using the multibib package
% \section{Publications}
%\nocitebook{book1,book2}
%\bibliographystylebook{plain}
%\bibliographybook{publications}                   % 'publications' is the name of a BibTeX file
%\nocitemisc{misc1,misc2,misc3}
%\bibliographystylemisc{plain}
%\bibliographymisc{publications}                   % 'publications' is the name of a BibTeX file

\end{document}
%-----       letter       ---------------------------------------------------------
% recipient data
\recipient{Larry Hodges, Ph.D.}{Co-Interim Chair, Human-Centered Computing\\Clemson University\\100 McAdams Hall\\Clemson, SC 29634}
\date{November 15, 2014}
\opening{Dear Dr. Hodges,}
\closing{Yours,}
\enclosure[Attached]{Curriculum Vit\ae{}}          % use an optional argument to use a string other than "Enclosure", or redefine \enclname
\makelettertitle

Lorem ipsum dolor sit amet, consectetur adipiscing elit. Duis ullamcorper neque sit amet lectus facilisis sed luctus nisl iaculis. Vivamus at neque arcu, sed tempor quam. Curabitur pharetra tincidunt tincidunt. Morbi volutpat feugiat mauris, quis tempor neque vehicula volutpat. Duis tristique justo vel massa fermentum accumsan. Mauris ante elit, feugiat vestibulum tempor eget, eleifend ac ipsum. Donec scelerisque lobortis ipsum eu vestibulum. Pellentesque vel massa at felis accumsan rhoncus.

Suspendisse commodo, massa eu congue tincidunt, elit mauris pellentesque orci, cursus tempor odio nisl euismod augue. Aliquam adipiscing nibh ut odio sodales et pulvinar tortor laoreet. Mauris a accumsan ligula. Class aptent taciti sociosqu ad litora torquent per conubia nostra, per inceptos himenaeos. Suspendisse vulputate sem vehicula ipsum varius nec tempus dui dapibus. Phasellus et est urna, ut auctor erat. Sed tincidunt odio id odio aliquam mattis. Donec sapien nulla, feugiat eget adipiscing sit amet, lacinia ut dolor. Phasellus tincidunt, leo a fringilla consectetur, felis diam aliquam urna, vitae aliquet lectus orci nec velit. Vivamus dapibus varius blandit.

Duis sit amet magna ante, at sodales diam. Aenean consectetur porta risus et sagittis. Ut interdum, enim varius pellentesque tincidunt, magna libero sodales tortor, ut fermentum nunc metus a ante. Vivamus odio leo, tincidunt eu luctus ut, sollicitudin sit amet metus. Nunc sed orci lectus. Ut sodales magna sed velit volutpat sit amet pulvinar diam venenatis.

Albert Einstein discovered that $e=mc^2$ in 1905.

\[ e=\lim_{n \to \infty} \left(1+\frac{1}{n}\right)^n \]

\makeletterclosing

%\clearpage\end{CJK*}                              % if you are typesetting your resume in Chinese using CJK; the \clearpage is required for fancyhdr to work correctly with CJK, though it kills the page numbering by making \lastpage undefined
\end{document}


%% end of file `template.tex'.
