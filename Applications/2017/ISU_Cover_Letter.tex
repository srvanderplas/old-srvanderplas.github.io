%%%%%%%%%%%%%%%%%%%%%%%%%%%%%%%%%%%%%%%%%%%%%%%%%%%%%%%%%%%%%%%%%%%%%
% Author: Susan VanderPlas
%
% This is an example of a complete CV using the 'moderncv' package
% and the 'timeline' package. For more information on those, please
% access:
% https://www.ctan.org/tex-archive/macros/latex/contrib/moderntimeline
% https://www.ctan.org/tex-archive/macros/latex/contrib/moderncv
%%%%%%%%%%%%%%%%%%%%%%%%%%%%%%%%%%%%%%%%%%%%%%%%%%%%%%%%%%%%%%%%%%%%%

\documentclass[12pt, letterpaper, sans]{moderncv}
\moderncvstyle{classic}
\moderncvcolor{blue}
\usepackage[utf8]{inputenc}
\usepackage[scale=0.85]{geometry}    % Width of the entire CV
\setlength{\hintscolumnwidth}{1.5in} % Width of the timeline on your left
\usepackage{pdfpages}
\usepackage{moderntimeline}
\usepackage{xpatch}
\usepackage{color, graphicx}
\usepackage[unicode]{hyperref}
\usepackage{xcolor}
\definecolor{link}{HTML}{3873B3}
\hypersetup{colorlinks, breaklinks,
            linkcolor=link,
            urlcolor=link,
            citecolor=link}

\tlmaxdates{2005}{2018}              % Beginning and start of your timeline

\newcommand{\cvreferencecolumn}[2]{%
  \cvitem[0.8em]{}{%
    \begin{minipage}[t]{\listdoubleitemmaincolumnwidth}#1\end{minipage}%
    \hfill%
    \begin{minipage}[t]{\listdoubleitemmaincolumnwidth}#2\end{minipage}%
    }%
}

\newcommand{\cvreference}[8]{%
    \textbf{#1}\newline% Name
    \ifthenelse{\equal{#2}{}}{}{\addresssymbol~#2\newline}%
    \ifthenelse{\equal{#3}{}}{}{#3\newline}%
    \ifthenelse{\equal{#4}{}}{}{#4\newline}%
    \ifthenelse{\equal{#5}{}}{}{#5\newline}%
    \ifthenelse{\equal{#6}{}}{}{\emailsymbol~\texttt{\href{mailto:#6}{\nolinkurl{#6}}}\newline}%
    \ifthenelse{\equal{#7}{}}{}{\phonesymbol~#7\newline}
    \ifthenelse{\equal{#8}{}}{}{\mobilephonesymbol~#8}}


% personal data

% Personal Information
\name{Susan}{VanderPlas}
\title{\emph{Cover Letter}}
\address{802 17th St.}{Auburn, NE 68305}{}
\phone[mobile]{515-509-6613}
%\phone[fixed]{+55~(11)~3091~2722}
\email{srvanderplas@gmail.com}                % optional, remove / comment the line if not wanted
\homepage{srvanderplas.com}                   % optional, remove / comment the line if not wanted
%\social[linkedin]{}                          % optional, remove / comment the line if not wanted
%\social[twitter]{srvanderplas}               % optional, remove / comment the line if not wanted
\social[github]{srvanderplas}                 % optional, remove / comment the line if not wanted
%\extrainfo{\emailsymbol \emaillink{}}

%\quote{Dubitando ad veritatem parvenimus - \emph{Cicerone}}

%----------------------------------------------------------------------------------
%            content
%----------------------------------------------------------------------------------
\begin{document}
% \makecvtitle
%-----       letter       ---------------------------------------------------------
% recipient data
\recipient{Dr. Max Morris}{Statistics Department\\Iowa State University\\Ames, IA 50010}
\date{November 1, 2017}
\opening{Dr. Morris,}
\closing{Yours,}
% \enclosure[Attached]{curriculum vit\ae{}}          % use an optional argument to use a string other than "Enclosure", or redefine \enclname
\makelettertitle

I write to apply for the Assistant Professor position in the Statistics Department at Iowa State University. I graduated from Iowa State University with a PhD in Statistics in May 2015. Since then, I have been working as a data scientist at Nebraska Public Power District (NPPD). During that time, I have been actively conducting research on my own, continuing my dissertation work on the perception of statistical graphics. My professional experience and my interdisciplinary research focus combine to make me a more effective teacher and collaborator. 

While working at NPPD, I designed and implemented a program to develop data science skills in current employees in diverse positions. While these employees will not become data scientists, the program's goal is to seed data-related skills across the company. I have worked on several statistical projects to improve efficiency and decision making, including an algorithm which predicts employee turnover, an R package to examine electrical load on transformers and breakers, and analyses which established that a nuclear plant was operating within regulations. Working with the IT department, I helped to design some of the infrastructure for data storage and analysis across the company; in the process, I learned how useful practical computing skills are for doing data science in industry.  

My doctoral research focused on the perception of statistical graphics, with the goal of designing graphics to effectively communicate statistical results. Research on graphical perception involves facets of many different disciplines, including psychology, statistics, human-computer interaction, computer science, and communication. In each of these fields, the research is focused on either extremely low-level effects, or very specific problems that are not applicable to most statistical graphics. My research integrates an understanding of the human perceptual system, methodology from cognitive psychology and human-computer interaction, and methods for graphical statistics in order to explore several facets of human perception that impact the way we interact with statistical graphics.

I have collaborated with researchers in bioinformatics, genetics, and engineering, assisting with statistical analyses as well as data visualization. My masters' research, a collaboration between bioinformatics and materials engineering, presented a nonparametric algorithm designed to mimic human perception of peaks in spectroscopy data with higher resolution than manual peak identification methods. During my doctoral research, I worked with soybean geneticists at the USDA to explore, analyze, and visualize populations of soybean sequence data; the massive amount of data are particularly challenging to visualize, as even heavily summarized data can overload the human visual system easily. 

Since my graduation in 2015, I have also been conducting research on a part-time basis. I conducted a follow-up study to ``Clusters beat Trend?! Testing Feature Hierarchy in Statistical Graphics" that is being prepared for publication, and I am currently working on a Bayesian adaptation of the analysis of statistical lineups. I am also collaborating with Dr. Heike Hofmann to explore the effectiveness of framed charts, which first appeared in the 1870 Statistical Atlas. 

In the future, I plan to investigate the perception of interactive graphics, creating guidelines for interactive graphics optimized for the human visual system. Visualizations which respond to user attention dynamically and incorporate motion through animation or other transitions have the potential to more intuitively communicate experimental results. The additional complexity of such graphics may reduce the amount of information that can be encoded, due to the increased demands on working memory and attention. As web-based interactive visualizations become more common, it is important that statisticians design graphics that are not only visually attractive, but also clearly communicate the overarching message in perceptually appropriate ways. 
%I also have plans to investigate the effect of domain-specific knowledge on static graph perception. While at NPPD, I noticed that several charts which violate best practices were effectively utilized by experts to make decisions; conversations with these individuals indicated that their expertise may have influenced the perception of these charts.

In addition to my research and industry experience, I have experience teaching both computational and introductory statistics classes. At NPPD, I worked with mid-career employees (from administrative assistants to economists), teaching both computational and statistical skills. For several years at Iowa State, I co-taught workshops on R programming as a resource for the university community (students, researchers, and local businesses), introducing the language and presenting topics including data visualization, formatting and arranging data for analysis, reproducible research with \texttt{knitr}, and linear models. I also have experience teaching undergraduate and graduate introductory statistics lab courses for engineering, bioinformatics, social science, and business students. Through all of these experiences, I have refined my approach to instruction, designing course materials which are thorough, provide chances for students to interact with the material in class in order to get useful feedback, and utilize humorous and topical examples to increase student engagement. 

The interdisciplinary nature of data science tends to attract students who have technical skills but often need to communicate their results to audiences with non-technical or disparate backgrounds. As I specialize in communicating such results using statistical graphics, I would be interested in developing or teaching a class which focuses on strategies for communication of technical results in a non-technical manner, including the use of graphics and charts to effectively convey information. I intend to incorporate real-world examples into this course which emphasize the importance of communicating actionable information to executives and other non-technical decision-makers. 

I am excited by the opportunity to return to the Iowa State statistics department. I greatly enjoyed my coursework and the opportunity to work on a diverse set of projects as a student, and I look forward to forming research collaborations which are equally diverse. If there are any additional materials I can provide, please feel free to contact me directly. Thank you for your consideration, and I look forward to hearing from you soon.


\makeletterclosing

\end{document}

