%%%%%%%%%%%%%%%%%%%%%%%%%%%%%%%%%%%%%%%%%%%%%%%%%%%%%%%%%%%%%%%%%%%%%
% Author: Susan VanderPlas
%
% This is an example of a complete CV using the 'moderncv' package
% and the 'timeline' package. For more information on those, please
% access:
% https://www.ctan.org/tex-archive/macros/latex/contrib/moderntimeline
% https://www.ctan.org/tex-archive/macros/latex/contrib/moderncv
%%%%%%%%%%%%%%%%%%%%%%%%%%%%%%%%%%%%%%%%%%%%%%%%%%%%%%%%%%%%%%%%%%%%%

\documentclass[12pt, letterpaper, sans]{moderncv}
\moderncvstyle{classic}
\moderncvcolor{blue}
\usepackage[utf8]{inputenc}
\usepackage[scale=0.85]{geometry}    % Width of the entire CV
\setlength{\hintscolumnwidth}{1.5in} % Width of the timeline on your left
\usepackage{pdfpages}
\usepackage{moderntimeline}
\usepackage{xpatch}
\usepackage{color, graphicx}
\usepackage[unicode]{hyperref}
\usepackage{xcolor}
\definecolor{link}{HTML}{3873B3}
\hypersetup{colorlinks, breaklinks,
            linkcolor=link,
            urlcolor=link,
            citecolor=link}

\tlmaxdates{2005}{2018}              % Beginning and start of your timeline

\newcommand{\cvreferencecolumn}[2]{%
  \cvitem[0.8em]{}{%
    \begin{minipage}[t]{\listdoubleitemmaincolumnwidth}#1\end{minipage}%
    \hfill%
    \begin{minipage}[t]{\listdoubleitemmaincolumnwidth}#2\end{minipage}%
    }%
}

\newcommand{\cvreference}[8]{%
    \textbf{#1}\newline% Name
    \ifthenelse{\equal{#2}{}}{}{\addresssymbol~#2\newline}%
    \ifthenelse{\equal{#3}{}}{}{#3\newline}%
    \ifthenelse{\equal{#4}{}}{}{#4\newline}%
    \ifthenelse{\equal{#5}{}}{}{#5\newline}%
    \ifthenelse{\equal{#6}{}}{}{\emailsymbol~\texttt{\href{mailto:#6}{\nolinkurl{#6}}}\newline}%
    \ifthenelse{\equal{#7}{}}{}{\phonesymbol~#7\newline}
    \ifthenelse{\equal{#8}{}}{}{\mobilephonesymbol~#8}}


% personal data

% Personal Information
\name{Susan}{VanderPlas}
\title{\emph{Curriculum Vitae}}
\address{802 17th St.}{Auburn, NE 68305}{}
\phone[mobile]{515-509-6613}
%\phone[fixed]{+55~(11)~3091~2722}
\email{srvanderplas@gmail.com}                % optional, remove / comment the line if not wanted
\homepage{srvanderplas.com}                   % optional, remove / comment the line if not wanted
%\social[linkedin]{}                          % optional, remove / comment the line if not wanted
%\social[twitter]{srvanderplas}               % optional, remove / comment the line if not wanted
\social[github]{srvanderplas}                 % optional, remove / comment the line if not wanted
%\extrainfo{\emailsymbol \emaillink{}}

%\quote{Dubitando ad veritatem parvenimus - \emph{Cicerone}}

%----------------------------------------------------------------------------------
%            content
%----------------------------------------------------------------------------------
\begin{document}
%-----       letter       ---------------------------------------------------------
% recipient data
\recipient{Krista Briley}{Industrial and Manufacturing System Engineering Department\\Iowa State University\\Ames, IA}
\date{September 25, 2017}
\opening{Dear Ms. Briley,}
\closing{Yours,}
% \enclosure[Attached]{curriculum vit\ae{}}          % use an optional argument to use a string other than "Enclosure", or redefine \enclname
\makelettertitle
% 
% 1. I would like to be considered for the position of [title copied from job ad] in [exact department name from job ad] at the [exact institution name from job ad]. I am an advanced doctoral candidate in [your department].
% I would recommend, “I am writing in application to the position…”  

I write to apply for the Industrial and Manufacturing Systems Engineering Department Faculty position in Data Science at Iowa State University. I graduated from Iowa State University with a PhD in Statistics in May 2015. Since then, I have been working as a statistical analyst (data scientist) at Nebraska Public Power District. Over the past two years, I have primarily worked with engineers to conduct data analyses to support engineering decisions, streamline NPPD's organizational practices, and make use of diverse sources of data to optimize business decisions. I believe this practical experience, combined with my statistical training and focus on data visualization make me an ideal candidate for the faculty position in your department.  

% 2. My doctoral project is a study of [cocktail party description]. Much of the research on this topic suggests that [characterize the literature as woefully inadequate]. But I [demonstrate, reveal, discover] that contrary to received wisdom, [your punch line].

% 3. To complete this research I have spent [X years] doing [fieldwork/lab work/archival work/statistical analysis]. I have traveled to [these cities or libraries], interviewed [X number of experts], created [original datasets/original compositions/original artwork].
% “This sentence should be followed by a paragraph with the story of your research process. Overwhelm the committee with the volume of artifacts you’ve studied, people you’ve talked to, time you have dedicated or places you’ve been.”
%  Deliver this information crisply, factually, in no more than 2 sentences.

% 4. I have completed [X] of [Y] chapters of my dissertation, and I have included two substantive chapters as part of my writing sample.
% This is ok, although telling them what chapters you have finished in the dissertation is less important than telling them a concrete defense date in the first sentence, in my view.  Talk is cheap, but a defense date doesn’t lie.

% 5. I have well-developed drafts of several other chapters, and expect to defend in [month, year]. OR Having defended in [month, year], I plan to [turn it into a book-length manuscript for a major scholarly press/select key chapters for publication in disciplinary journals].
% “If any of your committee members are unwilling to commit to even a season of the year for your defense date, or you don’t have two substantive chapters to submit to the hiring committee, it’s too early for you to be on the academic job market.”
My doctoral research focused on human perception of statistical graphics, with the ultimate goal of designing graphics to effectively communicate statistical results. Research on graphical perception is conducted in many different fields, including psychology, statistics, engineering, human-computer interaction, and communication. In each of these fields, though, the research focuses either on extremely low-level effects, or very specific problems that are not applicable to most statistical graphics. Integrating an understanding of human perception and methodology from cognitive psychology and human-computer interaction in an interdisciplinary approach to this problem, I explored several facets of human perception that meaningfully impact the way we interact with statistical graphics. My dissertation covers perceptual distortions that impact variance estimation along nonlinear trend lines (the ``sine illusion"), the influence of visual ability on graphical inference, and an exploration of the graphical features which are most visually compelling. Papers detailing the results of my dissertation research have been published in the Journal of Computational and Graphical Statistics, and Transactions on Visualization and Computer Graphics, and follow-up studies are currently being prepared for publication. 

After completion of my doctorate, I took a position at Nebraska Public Power District, first working directly for Cooper Nuclear Station and then reporting to corporate in Columbus, NE. My work at NPPD has provided me with considerable insight into the obstacles data scientists face, as well as the process of implementing a data science program in a business which is attempting to adapt to a rapidly changing economic situation. I have collaborated with engineers and managers, working on projects encompassing disaster preparedness, electrical demand prediction, industrial outage planning, and human resource optimization. 

Additionally, I have collaborated with researchers in bioinformatics, genetics, and engineering, assisting with statistical analyses as well as data visualization. My masters' research, a collaboration between bioinformatics and materials engineering, presented a nonparametric algorithm designed to mimic human perception of peaks in spectroscopy data with higher resolution than manual peak identification methods. While completing my doctorate, I worked with soybean geneticists at the USDA to explore, analyze, and visualize populations of soybean sequence data; the massive amount of data are particularly challenging to visualize, as even heavily summarized data can overload the human visual system easily. After graduation, I worked as a consultant for the USDA and the Iowa Soybean Association, designing interactive applets to communicate the results of statistical modeling to farmers as well as agronomists. 

In the future, I plan to investigate the perception of interactive graphics, creating guidelines for interactive graphics optimized for the human visual system. Visualizations which respond to user attention dynamically and incorporate motion through animation or other transitions have the potential to more intuitively communicate experimental results. The additional complexity of such graphics may reduce the amount of information that can be encoded, due to the increased demands on working memory and attention. As web-based interactive visualizations become more common, it is important that statisticians design graphics that are not only visually attractive, but also clearly communicate the overarching message in perceptually appropriate ways.

% 6. Although my primary area of research is [disciplinary keyword here], I have additional expertise in [another disciplinary keyword here] and am eager to teach in both areas. I have [taught/served as a teaching assistant] in courses about [A, B and C]. In the next few years, I hope to develop courses in [X and Y].
% Of course you must describe teaching competencies, but don’t do it with vague claims and emotion-talk.  Eager?  Hope?  How does that help us?  Again, 250 other people will resting their cases on exactly the same feelings. Give us facts and specifics instead.

In addition to my research, I have experience teaching both computational and introductory statistics classes. For several years, I co-taught workshops on R programming as a resource for the Iowa State community (students, researchers, and local businesses), introducing the language and presenting advanced topics including data visualization, formatting and arranging data for analysis, linear models. Extending this series, I designed workshops using new software packages, such as \texttt{knitr} for reproducible research and RStudio's \texttt{Shiny} package for creating interactive web applets. I have taught these classes while at NPPD as well, and expanded the courses into a Business Intelligence Embedded Agent program designed to develop data science skills within various departments at the company. I also have experience teaching undergraduate and graduate introductory statistics lab courses for engineering, bioinformatics, social science, and business students. In both settings, I utilize frequent examples which allowed students to independently apply the course material in a guided setting, reinforcing students' understanding of the material and providing opportunities for self-assessment and feedback. 

% 7. For the most part, my approach to research is through [social science or humanistic method keyword here], and I would be interested in developing a methods class on this approach to research.
The interdisciplinary nature of data science is likely to attract a group of students who have technical skills but often need to communicate their results to audiences with non-technical or disparate backgrounds. As I specialize in communicating such results using statistical graphics, I would be interested in developing or teaching a class which focuses on strategies for communication of technical results in a non-technical manner, including the use of graphics and charts to effectively convey information. 

% 8. Although I have been focused on my graduate research for several years, I have been actively involved in conversations with [scholars in the department you are applying to, or scholars at other universities/professional associations/conferences/other disciplines].
% “This can be the one paragraph about service, highlighting conferences you’ve attended, workshops you’ve organized, and other ways you’ve supported your discipline.”
% The key here is conferences. Those are peer-reviewed and your attendance at them will set your record apart.  Never hang your hat on service.

% covered in the research and teaching sections

% 9. In the next few years, I hope to be able to investigate [reasonably related problems or questions].
% Everybody needs a second project.  However, please don’t articulate it in vague aspirational feeling language like “hope.”  Are you really that unsure?  That doesn’t inspire confidence.  And while we’re on the subject, don’t try, attempt, endeavor, or seek, either.  Read my post: Do. Or Do Not. There Is No Try.

% covered earlier

% 10. I am interested in this post for a variety of reasons: [something about the character of the department/university/community/city].
I enjoyed living in Ames during my graduate training, and while I appreciate the experience I've gained working in industry, I am eager to return to academia (and to Ames). I enjoy working with engineers to solve problems and to more fully utilize data, and the idea of working in an engineering department while doing data science is very exciting. If there are any additional materials I can provide, please feel free to contact me directly or look through my research and current projects on github. Thank you for your consideration, and I look forward to hearing from you soon.  

% 11. Because of my graduate training, my doctoral research, and my teaching [experience/interests], I am uniquely qualified for this job.
%  Connect your achievements and record with the position without recourse to wheedling claims.
\vspace{1cm}
\makeletterclosing

%\clearpage\end{CJK*}                              % if you are typesetting your resume in Chinese using CJK; the \clearpage is required for fancyhdr to work correctly with CJK, though it kills the page numbering by making \lastpage undefined
\end{document}


%% end of file `template.tex'.
