%% start of file `template.tex'.
%% Copyright 2006-2013 Xavier Danaux (xdanaux@gmail.com).
%
% This work may be distributed and/or modified under the
% conditions of the LaTeX Project Public License version 1.3c,
% available at http://www.latex-project.org/lppl/.


\documentclass[11pt,letterpaper,sans,unicode]{moderncv}        % possible options include font size ('10pt', '11pt' and '12pt'), paper size ('a4paper', 'letterpaper', 'a5paper', 'legalpaper', 'executivepaper' and 'landscape') and font family ('sans' and 'roman')

\usepackage[nosort]{cite}
\usepackage{xstring} % bold individual name in bib entries
% \usepackage{biblatex} % separate bibliographies
\nocite{*}
\def\FormatMaidenName#1{%
  \IfSubStr{#1}{Koons}{\textbf{#1}}{#1}%
}
\def\FormatName#1{%
  \IfSubStr{#1}{VanderPlas}{\textbf{#1}}{\FormatMaidenName{#1}}%
}
% Color combination: 0099cc, ccffcc, 66ccff, 003399

% character encoding
\usepackage[utf8]{inputenc}                       % if you are not using xelatex ou lualatex, replace by the encoding you are using

% moderncv themes
\moderncvstyle{classic}                             % style options are 'casual' (default), 'classic', 'oldstyle' and 'banking'
\moderncvcolor{blue}                               % color options 'blue' (default), 'orange', 'green', 'red', 'purple', 'grey' and 'black'
%\renewcommand{\familydefault}{\sfdefault}         % to set the default font; use '\sfdefault' for the default sans serif font, '\rmdefault' for the default roman one, or any tex font name
%\nopagenumbers{}                                  % uncomment to suppress automatic page numbering for CVs longer than one page


%\usepackage{CJKutf8}                              % if you need to use CJK to typeset your resume in Chinese, Japanese or Korean

% adjust the page margins
\usepackage[scale=0.75]{geometry}
\setlength{\hintscolumnwidth}{2.3cm}                % if you want to change the width of the column with the dates
%\setlength{\makecvtitlenamewidth}{10cm}           % for the 'classic' style, if you want to force the width allocated to your name and avoid line breaks. be careful though, the length is normally calculated to avoid any overlap with your personal info; use this at your own typographical risks...

% Timeline
\usepackage{moderntimeline}
\tlmaxdates{2005}{2015} % Set the scale
\tlwidth{0.8ex} % Set the line width - space under the top label is 1pt more
\tltext{\tiny} % set label text size

\usepackage[unicode]{hyperref}
\usepackage{xcolor}
\definecolor{link}{HTML}{3873B3}
\hypersetup{colorlinks, breaklinks, 
            linkcolor=link, 
            urlcolor=link, 
            citecolor=link}
            
% personal data
\name{Susan}{VanderPlas}
% \title{Resumé title}                               % optional, remove / comment the line if not wanted
\address{802 17th St.}{Auburn, NE 68305}{}% optional, remove / comment the line if not wanted; the "postcode city" and "country" arguments can be omitted or provided empty
\phone[mobile]{(515) 509-6613}                   % optional, remove / comment the line if not wanted; the optional "type" of the phone can be "mobile" (default), "fixed" or "fax"
% \phone[fixed]{+2~(345)~678~901}
% \phone[fax]{+3~(456)~789~012}
\email{srvanderplas@gmail.com}                               % optional, remove / comment the line if not wanted
% \homepage{www.srvanderplas.github.io}                         % optional, remove / comment the line if not wanted
% \social[linkedin]{john.doe}                        % optional, remove / comment the line if not wanted
% \social[twitter]{jdoe}                             % optional, remove / comment the line if not wanted
\social[github]{srvanderplas}                     % optional, remove / comment the line if not wanted
% \extrainfo{additional information}                 % optional, remove / comment the line if not wanted
% \photo[64pt][0.4pt]{picture}                       % optional, remove / comment the line if not wanted; '64pt' is the height the picture must be resized to, 0.4pt is the thickness of the frame around it (put it to 0pt for no frame) and 'picture' is the name of the picture file
% \quote{Some quote}                                 % optional, remove / comment the line if not wanted

% to show numerical labels in the bibliography (default is to show no labels); only useful if you make citations in your resume
%\makeatletter
%\renewcommand*{\bibliographyitemlabel}{\@biblabel{\arabic{enumiv}}}
%\makeatother
%\renewcommand*{\bibliographyitemlabel}{[\arabic{enumiv}]}% CONSIDER REPLACING THE ABOVE BY THIS

% bibliography with mutiple entries
% \usepackage{multibib}
% \newcites{papers,presentations}{{Publications},{Presentations}}


%----------------------------------------------------------------------------------
%            content
%----------------------------------------------------------------------------------
\begin{document}
%-----       letter       ---------------------------------------------------------
% recipient data
\recipient{Dr. Shunpu Zhang}{Statistics Department\\University of Nebraska\\Lincoln, NE}
\date{November 12, 2014}
\opening{Dear Dr. Zhang,}
\closing{Yours,}
% \enclosure[Attached]{curriculum vit\ae{}}          % use an optional argument to use a string other than "Enclosure", or redefine \enclname
\makelettertitle
% 
% 1. I would like to be considered for the position of [title copied from job ad] in [exact department name from job ad] at the [exact institution name from job ad]. I am an advanced doctoral candidate in [your department].
% I would recommend, “I am writing in application to the position…”  

I write to apply for the Assistant Professor position in the Statistics Department at UNL. I am a PhD candidate in the Statistics Department at Iowa State, and am defending my dissertation in February 2015. My computing skills, experience in bioinformatics and social science, and research focus would nicely complement research in the department. 

% 2. My doctoral project is a study of [cocktail party description]. Much of the research on this topic suggests that [characterize the literature as woefully inadequate]. But I [demonstrate, reveal, discover] that contrary to received wisdom, [your punch line].

% 3. To complete this research I have spent [X years] doing [fieldwork/lab work/archival work/statistical analysis]. I have traveled to [these cities or libraries], interviewed [X number of experts], created [original datasets/original compositions/original artwork].
% “This sentence should be followed by a paragraph with the story of your research process. Overwhelm the committee with the volume of artifacts you’ve studied, people you’ve talked to, time you have dedicated or places you’ve been.”
%  Deliver this information crisply, factually, in no more than 2 sentences.

% 4. I have completed [X] of [Y] chapters of my dissertation, and I have included two substantive chapters as part of my writing sample.
% This is ok, although telling them what chapters you have finished in the dissertation is less important than telling them a concrete defense date in the first sentence, in my view.  Talk is cheap, but a defense date doesn’t lie.

% 5. I have well-developed drafts of several other chapters, and expect to defend in [month, year]. OR Having defended in [month, year], I plan to [turn it into a book-length manuscript for a major scholarly press/select key chapters for publication in disciplinary journals].
% “If any of your committee members are unwilling to commit to even a season of the year for your defense date, or you don’t have two substantive chapters to submit to the hiring committee, it’s too early for you to be on the academic job market.”
My doctoral research focuses on human perception of statistical graphics, with the ultimate goal of designing graphics to effectively communicate statistical results. Research on graphical perception is conducted in many different fields, including psychology, statistics, human-computer interaction, and communication. In each of these fields, though, the research focuses either on extremely low-level effects, or very specific problems that are not applicable to most statistical graphics. Integrating an understanding of human perception and methodology from cognitive psychology and human-computer interaction in an interdisciplinary approach to this problem, I am exploring several facets of human perception that meaningfully impact the way we interact with statistical graphics. My dissertation covers perceptual distortions that impact variance estimation along nonlinear trend lines (the ``sine illusion"), the influence of visual ability on graphical inference, and which features of a graph are most visually compelling. Two experiments examining the sine illusion were presented at JSM in 2013 and 2014, and one article has been accepted for publication in JCGS; another is nearly ready for submission. A paper detailing the relationship between visual ability and graphical inference is being prepared for publication as well, and I expect that the final study will be ready for publication review by March 2015. 

Additionally, I collaborate with researchers in bioinformatics, genetics, and engineering, assisting with statistical analyses as well as data visualization. My masters' research, a collaboration between bioinformatics and materials engineering, presented a nonparametric peak detection algorithm designed for use in spectroscopy data and explored Bayesian regression models for mass spectra. Currently, I am working with a group of soybean geneticists at UNL, Iowa State, and the USDA to explore, analyze, and visualize populations of soybean sequence data. This project is challenging on several fronts: it is extremely large, there are many separate analysis approaches, and even analysis results are difficult to visualize in context, because of the length of genetic code.

In the future, I plan to investigate the perception of interactive graphics, creating guidelines for interactive graphics optimized for the human visual system. Visualizations which respond to user attention dynamically and incorporate motion through animation or other transitions have the potential to more intuitively communicate results, particularly for large datasets which overwhelm static graphics. However, the additional complexity of interactive graphics may reduce the amount of information that can be encoded, due to the increased demands on working memory and attention. As web-based interactive visualizations become more common, it is important that statisticians design graphics that are not only visually attractive, but also clearly communicate the overarching message in perceptually appropriate ways.

% 6. Although my primary area of research is [disciplinary keyword here], I have additional expertise in [another disciplinary keyword here] and am eager to teach in both areas. I have [taught/served as a teaching assistant] in courses about [A, B and C]. In the next few years, I hope to develop courses in [X and Y].
% Of course you must describe teaching competencies, but don’t do it with vague claims and emotion-talk.  Eager?  Hope?  How does that help us?  Again, 250 other people will resting their cases on exactly the same feelings. Give us facts and specifics instead.
In addition to my research, I have experience teaching both computational and introductory statistics classes. For several years, I have co-taught workshops on R programming as a resource for the Iowa State community (students, researchers, and local businesses), introducing the language and presenting advanced topics including data visualization, formatting and arranging data for analysis, and linear models. Extending this series, I designed workshops using new software packages, such as \texttt{knitr} for reproducible research and Rstudio's \texttt{Shiny} package for creating interactive web applets. I also have experience teaching undergraduate and graduate introductory statistics lab courses for engineering, bioinformatics, social science, and business students. In both settings, I utilize frequent examples which allow students to independently apply the course material, reinforcing their understanding of the material and providing opportunities for self-assessment and feedback. 

% 7. For the most part, my approach to research is through [social science or humanistic method keyword here], and I would be interested in developing a methods class on this approach to research.
The interdisciplinary nature of applied statistics research often attracts students with technical skills who need to communicate results to non-technical audiences. As I specialize in communicating such results using statistical graphics, I would be interested in developing or teaching a class which focuses on strategies for communication of technical results in a non-technical manner, including the use of graphics and charts to effectively convey information. 

% 8. Although I have been focused on my graduate research for several years, I have been actively involved in conversations with [scholars in the department you are applying to, or scholars at other universities/professional associations/conferences/other disciplines].
% “This can be the one paragraph about service, highlighting conferences you’ve attended, workshops you’ve organized, and other ways you’ve supported your discipline.”
% The key here is conferences. Those are peer-reviewed and your attendance at them will set your record apart.  Never hang your hat on service.

% covered in the research and teaching sections

% 9. In the next few years, I hope to be able to investigate [reasonably related problems or questions].
% Everybody needs a second project.  However, please don’t articulate it in vague aspirational feeling language like “hope.”  Are you really that unsure?  That doesn’t inspire confidence.  And while we’re on the subject, don’t try, attempt, endeavor, or seek, either.  Read my post: Do. Or Do Not. There Is No Try.

% covered earlier

% 10. I am interested in this post for a variety of reasons: [something about the character of the department/university/community/city].
I had the opportunity to live on-campus at UNL for a summer as an undergraduate and very much enjoyed the school and the surrounding area. As a department, UNL seems to be very similar to Iowa State, focusing on applications of statistics while ensuring that new techniques are well-grounded in theory. If there are any additional materials I can provide, please feel free to contact me directly or look through my research and current projects on github. Thank you for your consideration, and I look forward to hearing from you soon.  

% 11. Because of my graduate training, my doctoral research, and my teaching [experience/interests], I am uniquely qualified for this job.
%  Connect your achievements and record with the position without recourse to wheedling claims.
\vspace{1cm}
\makeletterclosing

%\clearpage\end{CJK*}                              % if you are typesetting your resume in Chinese using CJK; the \clearpage is required for fancyhdr to work correctly with CJK, though it kills the page numbering by making \lastpage undefined
\end{document}


%% end of file `template.tex'.
