%%%%%%%%%%%%%%%%%%%%%%%%%%%%%%%%%%%%%%%%%%%%%%%%%%%%%%%%%%%%%%%%%%%%%
% Author: Susan VanderPlas
%
% This is an example of a complete CV using the 'moderncv' package
% and the 'timeline' package. For more information on those, please
% access:
% https://www.ctan.org/tex-archive/macros/latex/contrib/moderntimeline
% https://www.ctan.org/tex-archive/macros/latex/contrib/moderncv
%%%%%%%%%%%%%%%%%%%%%%%%%%%%%%%%%%%%%%%%%%%%%%%%%%%%%%%%%%%%%%%%%%%%%

\documentclass[12pt, letterpaper, sans]{moderncv}
\moderncvstyle{classic}
\moderncvcolor{blue}
\usepackage[utf8]{inputenc}
\usepackage[scale=0.85]{geometry}    % Width of the entire CV
\setlength{\hintscolumnwidth}{1.5in} % Width of the timeline on your left
\usepackage{pdfpages}
\usepackage{moderntimeline}
\usepackage{xpatch}
\usepackage{color, graphicx}
\usepackage[unicode]{hyperref}
\usepackage{xcolor}
\definecolor{link}{HTML}{3873B3}
\hypersetup{colorlinks, breaklinks,
            linkcolor=link,
            urlcolor=link,
            citecolor=link}

\tlmaxdates{2005}{2018}              % Beginning and start of your timeline

\newcommand{\cvreferencecolumn}[2]{%
  \cvitem[0.8em]{}{%
    \begin{minipage}[t]{\listdoubleitemmaincolumnwidth}#1\end{minipage}%
    \hfill%
    \begin{minipage}[t]{\listdoubleitemmaincolumnwidth}#2\end{minipage}%
    }%
}

\newcommand{\cvreference}[8]{%
    \textbf{#1}\newline% Name
    \ifthenelse{\equal{#2}{}}{}{\addresssymbol~#2\newline}%
    \ifthenelse{\equal{#3}{}}{}{#3\newline}%
    \ifthenelse{\equal{#4}{}}{}{#4\newline}%
    \ifthenelse{\equal{#5}{}}{}{#5\newline}%
    \ifthenelse{\equal{#6}{}}{}{\emailsymbol~\texttt{\href{mailto:#6}{\nolinkurl{#6}}}\newline}%
    \ifthenelse{\equal{#7}{}}{}{\phonesymbol~#7\newline}
    \ifthenelse{\equal{#8}{}}{}{\mobilephonesymbol~#8}}


% personal data

% Personal Information
\name{Susan}{VanderPlas}
\title{\emph{Cover Letter}}
\address{802 17th St.}{Auburn, NE 68305}{}
\phone[mobile]{515-509-6613}
%\phone[fixed]{+55~(11)~3091~2722}
\email{srvanderplas@gmail.com}                % optional, remove / comment the line if not wanted
% \homepage{srvanderplas.com}                   % optional, remove / comment the line if not wanted
%\social[linkedin]{}                          % optional, remove / comment the line if not wanted
%\social[twitter]{srvanderplas}               % optional, remove / comment the line if not wanted
\social[github]{srvanderplas}                 % optional, remove / comment the line if not wanted
%\extrainfo{\emailsymbol \emaillink{}}

%\quote{Dubitando ad veritatem parvenimus - \emph{Cicerone}}

%----------------------------------------------------------------------------------
%            content
%----------------------------------------------------------------------------------
\begin{document}
% \makecvtitle
%-----       letter       ---------------------------------------------------------
% recipient data
\recipient{Dan Nordman}{Statistics Department\\Iowa State University\\Ames, IA 50010}
\date{November 15, 2018}
\opening{Dr. Nordman,}
\closing{Yours,}
% \enclosure[Attached]{curriculum vit\ae{}}          % use an optional argument to use a string other than "Enclosure", or redefine \enclname
\makelettertitle

I write to apply for the assistant professor in forensic statistics position associated with the Center for Statistical Applications in Forensic Evidence. As you know, I am currently working for CSAFE as a research assistant professor. During my time at CSAFE, I have used my skills in statistics, data science, and machine learning on footwear, bullet, and fingerprint data. One of the most interesting parts of forensics for me is that we are in some ways trying to mimic human perception using algorithms, but at the same time, we must make these algorithms comprehensible by forensic examiners. I feel that my experience and training make me an ideal candidate for this position in part because I have worked with both sides of this dichotomy, not only during my current position, but also during my graduate work and time in industry. 

My current research at CSAFE is relatively diverse, but a common thread is a focus on leveraging the advantages of human perception and  statistical and computational algorithms, so that computers can be used in situations where repeatable, auditable results are required, even when completing tasks where humans have the advantage. I am working with a graduate student to build a neural network which can emulate human perception in order to identify class characteristics, such as geometric shapes, in shoe outsole designs. These characteristics are commonly used to identify shoe models from crime scene prints, and in most cases, forensic examiners do not proceed beyond class characteristic identifications. The neural network architecture is based on the architecture of the human visual cortex, and thus far, the neural network succeeds and fails in the same places that human perception does. This technique will be easily applicable to bullets (identifying tank rash and other areas where the matching algorithm underperforms) and fingerprint minutiae as well. I am also working on projects for resampling land engraved area signatures, which will allow characterization of the distribution of various features used for matching under known non-match and known match conditions. Both of these approaches to forensics are focused (in different ways) on using algorithms to mimic the process examiners use, which allows for auditable, repeatable results whose error rates can be determined relatively easily. The final project I am working on involves determining how jurors assess statistical charts when they are presented as evidence. We must be able to communicate statistical results effectively to juries, but also to prosecutors, lawyers, examiners, and judges in order for the statistics we develop to make lasting change. 

Prior to my work at CSAFE, I spent time in industry, where I helped to develop a data science training program for internal use at Nebraska Public Power District. This program was designed in a way similar to a flipped classroom, where online courses were used to teach concepts, and group meetings reinforced these concepts and demonstrated their application to real-world problems. My time at NPPD not only gave me some practical experience in data science applied to industry, but I also learned how to communicate statistical ideas to engineers, accountants, and even 

My doctoral research focused on the perception of statistical graphics, with the goal of designing graphics to effectively communicate statistical results. Research on graphical perception involves facets of many different disciplines, including psychology, statistics, human-computer interaction, computer science, and communication. In each of these fields, the research is focused on either extremely low-level effects, or very specific problems that are not applicable to most statistical graphics. During my doctoral training, I integrated an understanding of the human perceptual system, methodology from cognitive psychology and human-computer interaction, and methods for graphical statistics to explore several facets of human perception that impact the way we interact with statistical graphics. I have published papers in the Journal of Computational and Graphical Statistics (JCGS), IEEE Transactions on Visualization and Computer Graphics, and the Journal of Statistical Software. As a result of the research I conducted, I have a strong commitment to effectively communicating the results of statistical analyses in a visual domain which is accessible to statisticians and non-statisticians alike. 

Since graduation, I have been working as a statistical analyst at Nebraska Public Power District (NPPD), though a better job title would likely be ``data scientist". In addition to statistical analyses, I have developed and enhanced my skills in IT, server administration, database design and manipulation, data cleaning, data wrangling, and effective communication of results to decision-makers within the organization. In several instances, I have been tasked with answering questions from federal regulators, which requires an extra degree of precision and care. While I have not worked with forensic data, I have worked with data from a nuclear reactor core; I know very well that there are far-reaching consequences when an analysis is not conducted correctly. 

In addition to my research and experience doing data science, I have experience teaching both computational and introductory statistics classes. At both Iowa State and NPPD, I taught workshops on R programming, helping professionals and students to leverage the power of R for their own research or analysis needs. I have also taught introductory statistics lab courses for engineering, bioinformatics, social science, and business students. While I understand that teaching is not a large component of a research faculty position, I would be able to teach classes in data visualization, introductory statistics, or statistical computing if necessary. 

I enjoy working on analyses that have an impact on society. In my current position, I work to make power cheaper and more reliable, but I would like to work on projects which have a wider impact. I am excited by the chance to develop statistics for forensic analyses that may eventually make the justice system more reliable and fair. If there is any additional information I can provide, please feel free to contact me directly. Thank you for your consideration, and I look forward to hearing from you soon. 


\makeletterclosing

\end{document}

