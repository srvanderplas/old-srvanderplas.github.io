%%%%%%%%%%%%%%%%%%%%%%%%%%%%%%%%%%%%%%%%%%%%%%%%%%%%%%%%%%%%%%%%%%%%%
% Author: Susan VanderPlas
%
% This is an example of a complete CV using the 'moderncv' package
% and the 'timeline' package. For more information on those, please
% access:
% https://www.ctan.org/tex-archive/macros/latex/contrib/moderntimeline
% https://www.ctan.org/tex-archive/macros/latex/contrib/moderncv
%%%%%%%%%%%%%%%%%%%%%%%%%%%%%%%%%%%%%%%%%%%%%%%%%%%%%%%%%%%%%%%%%%%%%

\documentclass[12pt, letterpaper, sans]{moderncv}
\moderncvstyle{classic}
\moderncvcolor{blue}
\usepackage[utf8]{inputenc}
\usepackage[scale=0.85]{geometry}    % Width of the entire CV
\setlength{\hintscolumnwidth}{1.5in} % Width of the timeline on your left
\usepackage{pdfpages}
\usepackage{moderntimeline}
\usepackage{xpatch}
\usepackage{color, graphicx}
\usepackage[unicode]{hyperref}
\usepackage{xcolor}
\definecolor{link}{HTML}{3873B3}
\hypersetup{colorlinks, breaklinks,
            linkcolor=link,
            urlcolor=link,
            citecolor=link}

\tlmaxdates{2005}{2018}              % Beginning and start of your timeline

\newcommand{\cvreferencecolumn}[2]{%
  \cvitem[0.8em]{}{%
    \begin{minipage}[t]{\listdoubleitemmaincolumnwidth}#1\end{minipage}%
    \hfill%
    \begin{minipage}[t]{\listdoubleitemmaincolumnwidth}#2\end{minipage}%
    }%
}

\newcommand{\cvreference}[8]{%
    \textbf{#1}\newline% Name
    \ifthenelse{\equal{#2}{}}{}{\addresssymbol~#2\newline}%
    \ifthenelse{\equal{#3}{}}{}{#3\newline}%
    \ifthenelse{\equal{#4}{}}{}{#4\newline}%
    \ifthenelse{\equal{#5}{}}{}{#5\newline}%
    \ifthenelse{\equal{#6}{}}{}{\emailsymbol~\texttt{\href{mailto:#6}{\nolinkurl{#6}}}\newline}%
    \ifthenelse{\equal{#7}{}}{}{\phonesymbol~#7\newline}
    \ifthenelse{\equal{#8}{}}{}{\mobilephonesymbol~#8}}


% personal data

% Personal Information
\name{Susan}{VanderPlas}
\title{\emph{Curriculum Vitae}}
\address{801 Onyx Cir}{Ames, IA 50010}{}
\phone[mobile]{515-509-6613}
%\phone[fixed]{+55~(11)~3091~2722}
% \email{srvanderplas@gmail.com}                % optional, remove / comment the line if not wanted
\email{srvander@iastate.edu}                % optional, remove / comment the line if not wanted
% \homepage{srvanderplas.com}                   % optional, remove / comment the line if not wanted
%\social[linkedin]{}                          % optional, remove / comment the line if not wanted
%\social[twitter]{srvanderplas}               % optional, remove / comment the line if not wanted
\social[github]{srvanderplas}                 % optional, remove / comment the line if not wanted
%\extrainfo{\emailsymbol \emaillink{}}


%\quote{Dubitando ad veritatem parvenimus - \emph{Cicerone}}

%----------------------------------------------------------------------------------
%            content
%----------------------------------------------------------------------------------
\begin{document}
% \makecvtitle
%-----       letter       ---------------------------------------------------------
% recipient data
\recipient{Dan Nordman}{Statistics Department\\Iowa State University\\Ames, IA 50010}
\date{November 15, 2018}
\opening{Dr. Nordman,}
\closing{Yours,}
% \enclosure[Attached]{curriculum vit\ae{}}          % use an optional argument to use a string other than "Enclosure", or redefine \enclname
\makelettertitle

I am writing to apply for the tenure-track faculty position in Forensics (posting \# 800196). As you know, I returned to Iowa State in 2018 to work with CSAFE, with the goal of leveraging my computational skills and experience working with large data sets to make the justice system more equitable, efficient, and unbiased. I am actively working on several projects in forensics, including shoes, bullets, and human factors. I love the research that I am doing, but I miss teaching and interacting with students, and would like to take a position that includes these responsibilities in addition to research. 

Shortly after I received my PhD, I took a job in industry, working as a data scientist for Nebraska Public Power District. I worked on many different predictive and inferential projects, assessing the impact of weather on power plant operations, machinery reliability, and identifying factors related to employee retention. I also helped to design an internal training program for data science/business intelligence, which led to an unexpected opportunity to teach statistics and data science skills to a diverse set of professionals, from engineers to administrative assistants. This program, which we named the Business Intelligence Embedded Agent program, gave me insight into the challenges companies face as they incorporate data-driven decision making into their core operations. I plan to incorporate some of the experiences I had working at NPPD into any data science classes I teach in the future, because I discovered that no matter how correct my statistical analysis was, the results only made an impact if I could effectively communicate the problem, analysis, and results to stakeholders and managers. Students also are more likely to engage with the course material when there are tangible, practical applications that are relevant to their interests - I can leverage my experience in industry to ground any theory in practice. 

Since working with CSAFE, I have found that the forensics community is similarly wary of statistics - they are interested in some of the recent advances in statistical forensics, but are wary when we suggest bringing these advances into the examiner's lab. At the same time, in almost every area of forensics I have worked on, there are essential problems that humans solve easily but which are extremely difficult to automate. As an example, I am working with a graduate student to fit convolutional neural networks, which mimic the brain's ability to easily identify features, applying them to the identification of footwear outsole features. In the next few months, I plan to apply the same tool to bullets, with the goal of determining whether a land engraved area scan is suitable for use in the bullet matching algorithm. In both of these projects, the combination of a human-like machine learning algorithm and statistical pattern matching techniques represents a fusion of the best features of man and machine. By emphasizing the similarities between the examiner's approach and the algorithm's approach, we can increase the acceptance and impact of new methods and catalyze changes in the forensic community.

My PhD research focused on data visualization, with the specific goal of understanding how people perceive graphics and deriving best practices from experimental data. While this has obviously shaped the way I approach statistics, I have also honed skills in simulation, software development, image statistics, and working with `big' data. At CSAFE, these skills are incredibly useful, because pattern evidence is typically stored in images, and can become quite large, but they are also important skills for data scientists. I look forward to teaching students how to work with large, complex, and messy data sets, as well as involving them in research projects to develop their skills on real-world data. 

I believe my experience as a data scientist and as research faculty in statistical forensics make me a good candidate for a tenure-track position. 


% I have collaborated on research in a number of different fields, including psychology, engineering, and genetics, but the one common thread through all of these projects has been my focus on communicating the results of statistical analyses using visual and graphical tools. I am a senior researcher on an NIJ grant for 2019 to explore the best ways to use ROC curves to communicate the strength of forensic evidence to examiners, lawyers, and jurors. I am also laying the groundwork for a human factors project that will be part of the CSAFE renewal proposal, examining how jurors evaluate statistical evidence. My PhD research focused on the perception of statistical graphics, 




\makeletterclosing

\end{document}

