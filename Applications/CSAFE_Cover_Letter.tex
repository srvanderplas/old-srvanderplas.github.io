%%%%%%%%%%%%%%%%%%%%%%%%%%%%%%%%%%%%%%%%%%%%%%%%%%%%%%%%%%%%%%%%%%%%%
% Author: Susan VanderPlas
%
% This is an example of a complete CV using the 'moderncv' package
% and the 'timeline' package. For more information on those, please
% access:
% https://www.ctan.org/tex-archive/macros/latex/contrib/moderntimeline
% https://www.ctan.org/tex-archive/macros/latex/contrib/moderncv
%%%%%%%%%%%%%%%%%%%%%%%%%%%%%%%%%%%%%%%%%%%%%%%%%%%%%%%%%%%%%%%%%%%%%

\documentclass[12pt, letterpaper, sans]{moderncv}
\moderncvstyle{classic}
\moderncvcolor{blue}
\usepackage[utf8]{inputenc}
\usepackage[scale=0.85]{geometry}    % Width of the entire CV
\setlength{\hintscolumnwidth}{1.5in} % Width of the timeline on your left
\usepackage{pdfpages}
\usepackage{moderntimeline}
\usepackage{xpatch}
\usepackage{color, graphicx}
\usepackage[unicode]{hyperref}
\usepackage{xcolor}
\definecolor{link}{HTML}{3873B3}
\hypersetup{colorlinks, breaklinks,
            linkcolor=link,
            urlcolor=link,
            citecolor=link}

\tlmaxdates{2005}{2018}              % Beginning and start of your timeline

\newcommand{\cvreferencecolumn}[2]{%
  \cvitem[0.8em]{}{%
    \begin{minipage}[t]{\listdoubleitemmaincolumnwidth}#1\end{minipage}%
    \hfill%
    \begin{minipage}[t]{\listdoubleitemmaincolumnwidth}#2\end{minipage}%
    }%
}

\newcommand{\cvreference}[8]{%
    \textbf{#1}\newline% Name
    \ifthenelse{\equal{#2}{}}{}{\addresssymbol~#2\newline}%
    \ifthenelse{\equal{#3}{}}{}{#3\newline}%
    \ifthenelse{\equal{#4}{}}{}{#4\newline}%
    \ifthenelse{\equal{#5}{}}{}{#5\newline}%
    \ifthenelse{\equal{#6}{}}{}{\emailsymbol~\texttt{\href{mailto:#6}{\nolinkurl{#6}}}\newline}%
    \ifthenelse{\equal{#7}{}}{}{\phonesymbol~#7\newline}
    \ifthenelse{\equal{#8}{}}{}{\mobilephonesymbol~#8}}


% personal data

% Personal Information
\name{Susan}{VanderPlas}
\title{\emph{Cover Letter}}
\address{802 17th St.}{Auburn, NE 68305}{}
\phone[mobile]{515-509-6613}
%\phone[fixed]{+55~(11)~3091~2722}
\email{srvanderplas@gmail.com}                % optional, remove / comment the line if not wanted
\homepage{srvanderplas.com}                   % optional, remove / comment the line if not wanted
%\social[linkedin]{}                          % optional, remove / comment the line if not wanted
%\social[twitter]{srvanderplas}               % optional, remove / comment the line if not wanted
\social[github]{srvanderplas}                 % optional, remove / comment the line if not wanted
%\extrainfo{\emailsymbol \emaillink{}}

%\quote{Dubitando ad veritatem parvenimus - \emph{Cicerone}}

%----------------------------------------------------------------------------------
%            content
%----------------------------------------------------------------------------------
\begin{document}
% \makecvtitle
%-----       letter       ---------------------------------------------------------
% recipient data
\recipient{Alicia Carriquiry}{CSAFE\\Iowa State University\\Ames, IA 50010}
\date{September 1, 2017}
\opening{Dr. Carriquiry,}
\closing{Yours,}
% \enclosure[Attached]{curriculum vit\ae{}}          % use an optional argument to use a string other than "Enclosure", or redefine \enclname
\makelettertitle

I write to apply for the research faculty position with CSAFE at Iowa State University. I love challenges which can be solved by integrating statistical algorithms and perceptual principles. The images which comprise most of the data used at CSAFE are well-suited to this approach, which I have used successfully on a wide range of data sources. The combination of perceptual principles and statistics formed the foundation of my doctoral research at Iowa State in 2015. Since I graduated, I have been working as a data scientist at Nebraska Public Power District while consulting and conducting visualization research in my free time. My experience in statistical computing, data visualization, and bioinformatics has prepared me to set up databases for efficient analysis and to use those data structures to conduct analyses on large data sets. I am excited about the opportunity to return to academia and use my skills to analyze forensic data. 

As a graduate student at Iowa State, I worked with a team of soybean researchers lead by Dr. Michelle Graham to analyze and visualize next-generation sequencing data from 79 lines of soybeans. The initial genetic data was hundreds of gigabytes in size; even after analysis and identification of interesting features, the results from the analysis were between 2 and 12 GB. This magnitude of data is particularly challenging to visualize, as even heavily summarized data can easily overload the visual system. To address this problem, I designed applets which displayed the results of the analysis, using a combination of pre-computation and dynamic filtration to reduce the amount of data displayed to a manageable amount and to lessen the computational burden on the server. I expect that the lessons I learned during this project will transfer nicely to the analysis of forensic images, as even though the data sources are different, the magnitude of the data sets are fairly similar. 

My doctoral research focused on the perception of statistical graphics, with the goal of designing graphics to effectively communicate statistical results. Research on graphical perception involves facets of many different disciplines, including psychology, statistics, human-computer interaction, computer science, and communication. In each of these fields, the research is focused on either extremely low-level effects, or very specific problems that are not applicable to most statistical graphics. During my doctoral training, I integrated an understanding of the human perceptual system, methodology from cognitive psychology and human-computer interaction, and methods for graphical statistics to explore several facets of human perception that impact the way we interact with statistical graphics. I have published papers in the Journal of Computational and Graphical Statistics (JCGS), IEEE Transactions on Visualization and Computer Graphics, and the Journal of Statistical Software. As a result of the research I conducted, I have a strong commitment to effectively communicating the results of statistical analyses in a visual domain which is accessible to statisticians and non-statisticians alike. 

Since graduation, I have been working as a statistical analyst at Nebraska Public Power District (NPPD), though a better job title would likely be ``data scientist". In addition to statistical analyses, I have developed and enhanced my skills in IT, server administration, database design and manipulation, data cleaning, data wrangling, and effective communication of results to decision-makers within the organization. In several instances, I have been tasked with answering questions from federal regulators, which requires an extra degree of precision and care. While I have not worked with forensic data, I have worked with data from a nuclear reactor core; I know very well that there are far-reaching consequences when an analysis is not conducted correctly. 

In addition to my research and experience doing data science, I have experience teaching both computational and introductory statistics classes. At both Iowa State and NPPD, I taught workshops on R programming, helping professionals and students to leverage the power of R for their own research or analysis needs. I have also taught introductory statistics lab courses for engineering, bioinformatics, social science, and business students. While I understand that teaching is not a large component of a research faculty position, I would be able to teach classes in data visualization, introductory statistics, or statistical computing if necessary. 

I enjoy working on analyses that have an impact on society. In my current position, I work to make power cheaper and more reliable, but I would like to work on projects which have a wider impact. I am excited by the chance to develop statistics for forensic analyses that may eventually make the justice system more reliable and fair. If there is any additional information I can provide, please feel free to contact me directly. Thank you for your consideration, and I look forward to hearing from you soon. 


\makeletterclosing

\end{document}

