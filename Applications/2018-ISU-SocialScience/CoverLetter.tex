%%%%%%%%%%%%%%%%%%%%%%%%%%%%%%%%%%%%%%%%%%%%%%%%%%%%%%%%%%%%%%%%%%%%%
% Author: Susan Vanderplas
%
% This is an example of a complete CV using the 'moderncv' package
% and the 'timeline' package. For more information on those, please
% access:
% https://www.ctan.org/tex-archive/macros/latex/contrib/moderntimeline
% https://www.ctan.org/tex-archive/macros/latex/contrib/moderncv
%%%%%%%%%%%%%%%%%%%%%%%%%%%%%%%%%%%%%%%%%%%%%%%%%%%%%%%%%%%%%%%%%%%%%

\documentclass[12pt, letterpaper, sans]{moderncv}
\moderncvstyle{classic}
\moderncvcolor{blue}
\usepackage[utf8]{inputenc}
\usepackage[scale=0.85]{geometry}    % Width of the entire CV
\setlength{\hintscolumnwidth}{1.5in} % Width of the timeline on your left
\usepackage{pdfpages}
\usepackage{moderntimeline}
\usepackage{xpatch}
\usepackage{color, graphicx}
\usepackage[unicode]{hyperref}
\usepackage{xcolor}
\definecolor{link}{HTML}{3873B3}
\hypersetup{colorlinks, breaklinks,
            linkcolor=link,
            urlcolor=link,
            citecolor=link}

\tlmaxdates{2005}{2018}              % Beginning and start of your timeline

\newcommand{\cvreferencecolumn}[2]{%
  \cvitem[0.8em]{}{%
    \begin{minipage}[t]{\listdoubleitemmaincolumnwidth}#1\end{minipage}%
    \hfill%
    \begin{minipage}[t]{\listdoubleitemmaincolumnwidth}#2\end{minipage}%
    }%
}

\newcommand{\cvreference}[8]{%
    \textbf{#1}\newline% Name
    \ifthenelse{\equal{#2}{}}{}{\addresssymbol~#2\newline}%
    \ifthenelse{\equal{#3}{}}{}{#3\newline}%
    \ifthenelse{\equal{#4}{}}{}{#4\newline}%
    \ifthenelse{\equal{#5}{}}{}{#5\newline}%
    \ifthenelse{\equal{#6}{}}{}{\emailsymbol~\texttt{\href{mailto:#6}{\nolinkurl{#6}}}\newline}%
    \ifthenelse{\equal{#7}{}}{}{\phonesymbol~#7\newline}
    \ifthenelse{\equal{#8}{}}{}{\mobilephonesymbol~#8}}

\makeatletter
\renewcommand*{\makeletterclosing}{
  \vspace{1em}\@closing\\%
  \hspace{-1em}\includegraphics{/home/susan/Pictures/Signature.png}\\% Insert signature
  {\bfseries \@firstname~\@lastname}%
  \ifthenelse{\isundefined{\@enclosure}}{}{%
    \\%
    \vfill%
    {\color{color2}\itshape\enclname: \@enclosure}}}
\makeatother

% personal data

% Personal Information
\name{Susan}{Vanderplas}
\title{\emph{Curriculum Vitae}}
\address{801 Onyx Cir}{Ames, IA 50010}{}
\phone[mobile]{515-509-6613}
%\phone[fixed]{+55~(11)~3091~2722}
% \email{srvanderplas@gmail.com}                % optional, remove / comment the line if not wanted
\email{srvander@iastate.edu}                % optional, remove / comment the line if not wanted
% \homepage{srvanderplas.com}                   % optional, remove / comment the line if not wanted
%\social[linkedin]{}                          % optional, remove / comment the line if not wanted
%\social[twitter]{srvanderplas}               % optional, remove / comment the line if not wanted
\social[github]{srvanderplas}                 % optional, remove / comment the line if not wanted
%\extrainfo{\emailsymbol \emaillink{}}


%\quote{Dubitando ad veritatem parvenimus - \emph{Cicerone}}

%----------------------------------------------------------------------------------
%            content
%----------------------------------------------------------------------------------
\begin{document}
% \makecvtitle
%-----       letter       ---------------------------------------------------------
% recipient data
\recipient{Petrutza Caragea}{Statistics Department\\Iowa State University\\Ames, IA 50011}
\date{November 15, 2018}
\opening{Dear Dr. Caragea,}
\closing{Yours,}
% \enclosure[Attached]{curriculum vit\ae{}}          % use an optional argument to use a string other than "Enclosure", or redefine \enclname
\makelettertitle

I am writing to apply for the tenure-track faculty position in statistics with a focus on social science (posting \# 800190). As you know, I returned to Iowa State in 2018 to work with CSAFE, with the goal of leveraging my computational skills and experience working with large data sets to make the justice system more equitable, efficient, and unbiased. I am actively working on several projects in forensics, and most of these projects are at the intersection of machine learning, psychology, computing, statistics, and the legal system. I love the research that I am doing, but I miss teaching and interacting with students and would like a position that includes these responsibilities in addition to research. 

Shortly after I received my Ph.D., I took a job as a statistical analyst for Nebraska Public Power District. I worked on many different predictive and inferential projects, assessing the impact of weather on power plant operations, evaluating machinery reliability, and leveraging data for decisions in human resources. I developed a model to predict employee turnover, leveraging machine learning to predict human behavior; on several occasions, this model predicted otherwise surprising resignations. I also designed an internal training program for data science/business intelligence, which led to an unexpected opportunity to teach statistics and data science skills to a diverse set of professionals, from engineers to administrative assistants. This program gave me insight into the challenges companies face as they incorporate data-driven decision making into their core operations, as well as the skill sets required for data scientists. I found that no matter how correct the analysis, the results only mattered if I could quickly and effectively communicate the problem, analysis, and results to stakeholders and managers. As a result, I plan to incorporate communication skills and computing skills useful in industry when I teach data science and statistics. 

Since working with CSAFE, I have found that forensic practitioners can be similarly wary of statistics - they are interested in some of our recent advances, but resist bringing these advances into the lab. Some of this caution is justified: in almost every area of forensics, there are tasks which humans do automatically, but which are hard for computers. Using algorithms designed to mimic the human approach to the problem, such as convolutional neural networks (CNNs) for image classification tasks, we can combine the strengths of man and machine. We can then increase the acceptance and impact of new methods by emphasizing the similarities between the examiner's approach and the algorithm's approach, catalyzing changes in the forensic community. I am currently working with a graduate student to apply CNNs to features of shoe treads, which will make automatic identification of shoe models a much easier task; we have plans to use the same technique to identify bullet scans which are low quality and not fit for analysis. These are classic machine learning problems, but incorporate knowledge of human memory, perception, and cognitive processes. In order for humans to benefit from machine learning, the insights from these models must be human-friendly, which requires an understanding of numerical cognition as well as statistical theory.

My Ph.D. research in graphics and visualization has obviously influenced my approach to statistics, in part because my undergraduate work was in cognitive psychology as well as applied mathematics. During my graduate work and time in industry, I developed skills and expertise in simulation, software development, image analysis, machine learning, and working with large data sets; all of these skills are useful in both applied statistics and data science. One of the biggest differences between data science and statistics seems to be the expectation that data scientists have not only expertise in statistical methods and analysis, but also that they have some ability to build software pipelines to transform messy data into meaningful results. One of my projects at CSAFE spans the data science continuum: the final product is an open source software package that encapsulates the data processing pipeline, but there is significant statistical work to validate the machine learning algorithms underpinning the process as well as the expected distribution of the results under different conditions. I am currently working on several publications related to the project: a validation of the pipeline on several case studies, a method for generating simulated input data using the threshold bootstrap, and a discussion of the importance of unit testing and continuous integration in software developed for use in forensics. In the social sciences there is a similar breadth of skill required for many research projects, from UI design and human factors to developing new methods for statistical analysis and packaging software to facilitate adoption of new methods.

% Teaching paragraph here
The main goal of data visualization research is to understand and communicate statistical information in an approachable and digestible manner. This focus provides an advantage when teaching, because I communicate with visual illustrations of statistical and computational concepts in addition to verbal explanations. While a graduate student in this department, I had the privilege to teach the lab portions of several graduate and undergraduate courses, as well as R programming seminars for the Iowa State community. After graduation, I taught data science to professionals in an industrial setting. In both of these situations, I used my research to inform my teaching and improve communication with students. 

I believe my experience as a data scientist and as research faculty in statistical forensics make me a good candidate for a tenure track position in the statistics department focused on social science. In particular, I have experience teaching and mentoring data science students, developing statistical software, and utilizing statistical and machine learning techniques in forensics, engineering, psychology, bioinformatics and agronomy. My research includes consideration of visual perception, memory, group dynamics, and numerical cognition; in many ways, I already have a foot in the world of social science. I enjoy working at CSAFE and being a faculty member in the statistics department, and I am ready for the additional responsibility of teaching courses and working more closely with students. Please let me know if I can provide any additional information to assist you with your decision, and thank you for your consideration.  


% I have collaborated on research in a number of different fields, including psychology, engineering, and genetics, but the one common thread through all of these projects has been my focus on communicating the results of statistical analyses using visual and graphical tools. I am a senior researcher on an NIJ grant for 2019 to explore the best ways to use ROC curves to communicate the strength of forensic evidence to examiners, lawyers, and jurors. I am also laying the groundwork for a human factors project that will be part of the CSAFE renewal proposal, examining how jurors evaluate statistical evidence. My Ph.D. research focused on the perception of statistical graphics, 




\makeletterclosing

\end{document}

